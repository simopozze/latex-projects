\documentclass[a4paper,12pt]{article}
\usepackage{hyperref}
\usepackage{fancyhdr}
\pagestyle{fancy}
\fancyhf{}
\lhead{Ministero dell’Istruzione, dell’Università e della Ricerca}
\rfoot{Page \thepage}

\author{Simone Pozzebon}
\date{}
\title{ESAME DI STATO DI ISTRUZIONE \\ SECONDARIA SUPERIORE}

\begin{document}
\maketitle
\newpage
\textbf{Indirizzo:}
\begin{itemize}
\item [] ITIA - INFORMATICA 
\item [] TELECOMUNICAZIONI ARTICOLAZIONE INFORMATICA
\end{itemize}

\textbf{Tema di:} INFORMATICA e SISTEMI E RETI \\ \\
\begin{center}
\textbf{Il candidato svolga la prima parte della prova e due tra i quesiti
proposti nella seconda parte.}
\end{center}

\section{\textbf{PRIMA PARTE}}
L' azienda "easyCharge" si occupa di fornire su tutto il territorio italiano
colonnine di ricarica per autovetture elettriche. 

L'utente puo' quindi accedere al servizio mediante una tessera reperibile
attraverso un portale web. L'utente necessita solamente di un indirizzo 
email valido o eventualmente una carta di pamento prepagata.

Una volta forniti i dati il portale richiede l'inserimento di un codice di 
verifica precedentemente inviato all'indirizzo email assegnato; se tutte le 
informazioni coincidono, l'utente verra' loggato nel sito. 

L'utente puo' dunque scegliere che tipo di contratto sottoscrivere con l'
azienda. Il \textit{premium} permette di avere accesso illimitato alle colonnine,
in qualunque momento e senza limiti di tempo, in cambio di un pagamento
mensile variabile in base al comune di installazione del servizio. \\
Il contratto \textit{non premium gratuito} limita l'accesso al servizio a sole
50 ore di ricarica settimanali. \\
Le operazioni di pagamento sono totalmente gestite e 
tutelate da PayPal.

Per avviare il servizio la colonnina necessita di un codice univoco per 
identificare l'utente, cio' avviene grazie a un lettore bar-code, l'utente 
trova il codice a barre scaricabile direttamente nella propria area personale nel 
portale web.

La distribuzione delle colonnine di ricarica sul territorio e' capillare, in
ogni comune l'azienda fornisce una postazione ogni 10 autovetture elettriche 
registrate alla motorizzazione.

Le informazioni di fatturazione e statistiche di utilizzo di ogni comune fanno
riferimento a un server regionale che, a sua volta elabora e fornisce 
i dati all'azienda.

Ogni colonnina in funzione (si considera tale quando eroga energia in seguito a
un' attivazione del servizio da parte di un \textit{utente registrato 
e verificato}) invia al server di riferimento un token (Id univoco di ogni 
postazione) ogni mezz'ora.\\ Questa operazione viene svolta per controllare
l'utilizzo medio di ogni colonnina ed evitare problemi di dispersione
energetica.

L'azienza fornisce anche un'applicazione mobile, disponibile solo nel
piano \textit{premium} che permette all'utente di controllare le proprie
statistiche di utilizzo e di avviare il servizio sfruttando la connessione NFC
del dispositivo (ove disponibile) senza ricorrere dunque al bar-code.

Il candidato analizzi la realtà di riferimento e, fatte le opportune ipotesi 
aggiuntive, individui una
soluzione che a suo motivato giudizio sia la più idonea 
per svilupparei seguenti punti: 

\begin{itemize}
\item Considerazioni e accorgimenti riguardo:
\begin{enumerate}
\item L'infrastruttura che mette in comunicazione le colonnine, eventualmente
mediante grafici e rappresentazioni concettuali;
\item Le decisioni hardware e software da attuare per garantire 
un livello il piu' elevato possibile di velocita' e sicurezza nelle transizioni
economiche per l'abilitazione del servizio, facendo particolare attenzione
alla tutela delle informazioni inserite sul portale web o sull'applicazione 
mobile;
\item Il sistema DBMS per la gestione delle informazioni dinamiche e statiche
nell'infrastruttura. Diagramma ER e Schema Logico;
\end{enumerate}
\item La codifica in linguaggio SQL delle seguenti interrogazioni:
\begin{enumerate}
\item L'elenco degli utenti premium presenti in un determinato comune;
\item Il conteggio delle colonnine in funzione in una determinata fascia oraria;
\item L'elenco di tutte le autovetture elettriche presenti sul territorio 
italiano al momento del censimento (1 colonnina per ogni 10 autovetture);
\end{enumerate}
\end{itemize}
\section{\textbf{SECONDA PARTE}} 
\begin{enumerate}
\item Il candidato e' stato assunto come sviluppatore da "easyCharge" per 
migliorare l'applicazione mobile. E' quindi richiesta l'integrazione di una 
funzionalita' per localizzare le colonnine in funzione nelle vicinanze (GPS
tracking) per visualizzare quelle con carico minore (minore quantitativo di
energia erogato in maggior tempo);
\item Si progettino il form di login, logout e sign up per il portale web;
\item Si progettino il form per la gestione delle transazioni online, integrando
le componenti necessarie per cifrare la comunicazione.
\end{enumerate}
\end{document}


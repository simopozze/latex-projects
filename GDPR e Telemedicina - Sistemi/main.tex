\documentclass[a4paper,12pt]{article}
\usepackage[hidelinks]{hyperref}

\author{Simone Pozzebon}
\title{GDPR - TELEMEDICINA}

\begin{document}
\maketitle
\tableofcontents
\begin{center}
    \href{https://slides.com/simopozze/telemedicina/}{LINK ALLA PRESENTAZIONE}
\end{center}
\newpage

\section{TELEMEDICINA}
\subsection{Introduzione}
Oggigiorno ci troviamo a dover fare i conti con nuove sfide, in questo momento piu' che mai
ci viene richiesto di trovare soluzioni ai problemi concreti che il Coronavirus ha portato
alla nostra quotidianita'. E' dunque possibile rimanere connessi in tempo reale con chiunque, abbiamo la possibilita' di lavorare 
e studiare come mai fatto prima.
 La crisi sanitaria ha contribuito inoltre allo sviluppo di un nuovo ambito tecnologico: \textit{la telemedicina}:
 
 \subsection{Definizione}
 \paragraph{La telemedicina e' la comunicazione dei dati medici a fini diagnostici e terapeutici ottenuti a distanza attraverso 
    supporti tecnologici"}

\subsection{Attuazione}
\begin{enumerate}
    \item Nel rapporto tra differenti ospedali circa lo scambio di referti in formato digitale. Velocizzando il processo e garantendo l'integrita' 
    dei documenti scambiati.
    \item Nel rapporto tra medici e pazienti malati cronici al fine di agevolare il monitoraggio delle risorse e dell'avanzare dell malattia. \\ Viene definito: \textit{telemonitoraggio}.
\end{enumerate}

\section{IL GDPR}
\subsection{Introduzione}
Questa nuova tecnica implica pero' un ingente interscambio di informazioni \textit{riservate} e \textit{sensibili}.
E' quindi necessaria una specifica tutela legale al fine di garantire correttamente con riservo transizione e stazionamento dei dati.

\subsection{Cenni Storici}
Ben prima delle attuali tecnologie telematiche il nostro paese si e' impegnato per la tutela dei dati sensibili.\\ Nel 1996 viene emanato
il \textit{Testo Unico In  Protezione Dei Dati}. \\

Esso porta una razionalizzazione e prim semplificazione delle norme esistenti. Successivamente a questo
viene stipulato nel 2003 \textit{Il Codice Della Privacy}, primo testo in collaborazione con la comunita' europea.
Lo step finale arriva nel 2016 con il GDPR 

\subsection{Definizione}
\textit{General Data Protection Regulation}, meglio conosciuto come testo unico europeo per la protezione dei dati personali.
Punto di forza e' infatti l'applicabilita' del manuale a tutti i paesi facenti parte della comunita' europea.
\\Esso si suddivide in 11 articoli, riguardanti i principi, i diritti e i doveri degli enti nelle attivita' telematiche e informatiche.

\subsection{I dati \textit{Sensibili}}
Gia' prima del GDPR per il trattamento dei dati \textit{sensibili} era necessario il consenso \textit{informato} ed \textit{esplicito} del diretto interessato.
Essi sono:
\begin{itemize}
    \item Origine etnica;
    \item Opininoni politiche e religiose;
    \item Appartenenza sindacale;
    \item Dati genetici;
    \item Dati sanitari e relativi all'orientamento sessuale;
\end{itemize}
\subsection{I dati \textit{Personali}}
Nel 2016 pero', in seguito alla sua introduzione, cambia la definizone stessa di \textit{dato personale}. Da quel momento in poi infatti: 
\paragraph{``I dati personali rappresentano tutte le informazioni riconducibili ad un singolo individuo"}

\begin{itemize}
    \item La categoria definita in precedenza \textit{dati sensibili};
    \item Ogni informazine anagrafica;
    \item Le informazioni giudiziarie;
    \item Le informazioni identificative telematiche (indirizzo IP,...);
    \item La posizione geografica ottenuta tramite GPS;
    \item I dati biometrici digiali (scansione delle impronte digiali, retina, ...);
\end{itemize}

\subsection{Garantire il GDPR in azienda}
In che modo e' quindi possibilie assicurare il pieno adempimento degli 11 articoli del testo unico?
Importante e' sottolineare fin da subito che OGNI soggetto imprenditoriale e' tenuto a conoscere tali normative. \\
Dal sito di E-Commerce all'artigiano di paese che condivide il portfolio online; unica distinzione si fa per aziende con un numero di dipendenti inferiore ai 250, per esse vi e' solo il ``consiglio'' e non l'obbligo di adeguamento. In senso generale qualora si trattino dati personali su larga scala bisogna:
\begin{enumerate}
    \item Nominare un \textit{RPD - Responsabile Protezione Dati}, una figura professionale formata per assistere il titolare in azienda 
    nel garantire l'adeguamento \\ iniziale e il successivo mantenimento delle normative;
    \item Tutelare la riservatezza dei dati applicando metodi crittografici;
    \item Strutturare sistemi di ripristino istantanei in caso di data breach e immediato report al \textit{Garante della Privacy} in caso di problemi. 
    Figura amministrativa indipendente istituita nel 1996 per assicurare il controllo e la protezione delle informazioni private del cittadino;
    \item Redigere un registro delle attivita' eseguite sui dati personali e su eventuali comportamenti ritenuti rischiosi;
\end{enumerate}
\subsection{Garantire il GDPR all'interno dell'infrastruttura IT}
Parlando nello specifico delle reti, all'interno di un'azienda, diventa quindi necessario per essere in linea con le normative europee affidarsi a partner certificati 
in grado di garantire \textit{Rating 4} secondo \textit{Standard ANSI/TIA 942-A}. \\ \\ 
Ogni struttura di supporto deve adeguarsi e garantire:
\begin{enumerate}
    \item Resilienza;
    \item Riservatezza;
    \item Sicurezza fisica e meccanica;
    \item Ridondanza elettrica e telematica;
    \item Ignifugazione
\end{enumerate}

Tutto certificato dal codice di condotta del \textit{CISPE - Cloud Infrastracture Service in Europe}

\subsection{Il GDPR nelle tecnologie \textit{Cloud}}
La normativa si esprime anche riguardo le tecnologie cloud, sia in campo business che extra aziendali. \\
Sono definite le linee guida riguardo:
\begin{enumerate}
    \item \textit{Cloud Backup}: backup \textit{automatizzati} con \textit{cifratura} del dato e trasmissione.
    \item \textit{DRAAS - Distaster Recovery As A Structure}: consente la replica delle informazioni in ambiente
    virtualizzato. Soddisfacendo quindi requisiti di \textit{resilienza}, \textit{ripristino} e \textit{Live Test}.
    \item \textit{Business Continuity}
    \begin{enumerate}
        \item \textit{Hardware}: cablaggio strutturato tra i Data Center sfruttando collegamenti in fibra ottica. Velocizzando
        in questo modo eventuali ripristini.
        \item \textit{Human}: i dipendenti sono continuamenti formati per seguire i protocolli in modo da essere in grado di affrontare
        ogni tipo di eventualita', oltreche' abili nel redarre il corretto report della situazione di crisi.
    \end{enumerate}
\end{enumerate}
\end{document}

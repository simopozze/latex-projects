\documentclass[a4paper, 12pt]{article}
\usepackage[utf8]{inputenc}
\usepackage[hidelinks]{hyperref}
\usepackage[italian]{babel}

\title{Business Plan - Analisi}
\author{Simone Pozzebon}
\date{\today}

\begin{document}
    \maketitle
    \tableofcontents
    \newpage    
    
    \section{Introduzione}
        Un business plan è il primo passo importante per tutti coloro che iniziano un'impresa. \\
        Aiuta a monitorare i progressi dell' azienda (o le loro mancanze). Esistono varie tipologie di piani aziendali per vari scopi e i migliori piani sono documenti in continua evoluzione che reagiscono con il mondo reale il più rapidamente possibile. 
        
        Se hai un'idea per una nuova azienda, un business plan ti aiuterà a decidere se la tua idea è fattibile.
        E’ sconsigliato quindi avviare un'azienda se c'è poca o nessuna possibilità di profitto,  una strategia aziendale ti aiuterà quindi a determinare le tue probabilità di successo.
        
        In alcuni casi, i liberi professionisti che si lanciano in nuove società non hanno nemmeno i  fondi necessari per far decollare la loro impresa, se c'è quindi la necessità di un determinato capitale iniziale, serve avere un piano aziendale da presentare ai potenziali investitori che dimostri come la nuova attività possa  avere successo. 
        
        Quando si tratta di creare dei resoconti per  la strategia utilizzata, gli imprenditori hanno molto margine di manovra, essi possono essere brevi e sintetici oppure lunghi e dettagliati e possono fornire tutti i dettagli che si ritengono utili.
        
    \section{Principali Componenti}
        \subsection{Analisi di Mercato}
            L'analisi di mercato rivelerà se esiste una domanda sufficiente per il tuo prodotto o servizio, interna al tuo mercato di riferimento.
            
            Se il mercato dovesse essere già saturo, il modello business dovrà essere cambiato o scartato.
        
        \subsection{Piano di Gestione}
            La strategia di gestione descrive in dettaglio la struttura aziendale, la gestione e le esigenze del personale.
            Quando l’azienda ha competenze di gestione e dipendenti uniche, si avrà bisogno di un piano per localizzare, reclutare e mantenere dipendenti qualificati.
        
        \subsection{Piano Operativo}
            Il piano operativo descrive i servizi, i macchinari, l'inventario e le esigenze di fornitura.
            Molte aziende mettono un premio sulla posizione e la connettività.
            Se l’azienda necessita di componenti o materie prime per fabbricare prodotti che saranno venduti ai consumatori, si dovrà anche guardare alle future catene di approvvigionamento.
            
        \subsection{Piano Finanziario}
        Il piano finanziario determinerà se il nuovo progetto di business possa essere o meno un successo.
        Se il finanziamento è necessario, la strategia finanziaria sancirà le possibilità di ottenere investimenti per start-up da banche, angel investors o venture capitalist sotto forma di finanziamento del debito.
        Si può avere una strategia di business brillante, così come eccellente marketing, gestione e strategie operative, ma se il piano finanziario rivela che il business non sarà abbastanza redditizio, il modello di business non è sostenibile, e non c'è alcun punto di forza per iniziare l'impresa.

\end{document}

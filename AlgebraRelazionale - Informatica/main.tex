\documentclass[a4paper, 12pt]{report}
\usepackage{hyperref}
\usepackage{mathabx}

\title{ALGEBRA RELAZIONALE - Informatica}
\author{Simone Pozzebon}
\date{}

\begin{document}
    \maketitle
    \tableofcontents
    \newpage
    \newpage

    \section{Definizione di Relazione}
    \begin{center}
        \textbf{RELAZIONE E' IL SOTTINSIEME DEL PRODOTTO \newline CARTESIANO ALL'INTERNO DELL'INSIEME
    A1} \newline  
        \[R \epsilon {a1, a2,...an} | a1 \epsilon A1 \vee i=1...n\] \newline
    \end{center}
    \begin{itemize}
        \item Ai = i-esimo dominio della relazione;
        \item n  = numero dei \textbf{domini} = grado della relazione;
        \item /R/ = \textbf{cardinalita'} della relazione = numero di n-uple di R
    \end{itemize}
    
    \subsection{OGNI ELEMENTO DI R - RELAZIONE}
    \begin{itemize}
        \item e' ELEMENTO
        \item e' ELEMENTO DI UN INSIEME
        \item e' ENTITA'
    \end{itemize}
    
    \begin{center}
        \begin{table}[h]
            \begin{tabular}{|c|c|c|ll}
            \cline{1-3}
            \textbf{ENTITA'} & \textbf{CODD / RELAZIONE}      & \textbf{TABELLA}                  &  &  \\ \cline{1-3}
            N. Attributi     & Grado relazione         & N.Colonne/ N. Campi del tracciato &  &  \\ \cline{1-3}
            Istanza          & N-upla                  & Riga/Record                       &  &  \\ \cline{1-3}
            N.Istanze        & Cardinalita'/N.Elementi & N.Righe/N.Record                  &  &  \\ \cline{1-3}
            \end{tabular}
            \end{table}
    \end{center}

    \section{VINCOLI DELLE N-UPLE}
    \begin{enumerate}
        \item \textbf{Stesso numero di elementi.} Le righe della tabella devono avere lo stesso numero di colonne. In caso contrario si inseriscono dati NULL;
        \item Per ogni elemento di n-upla \textbf{ci sono valori OMOGENEI, dello stesso DOMINIO}
        \item R \textbf{ NON PUO' AVERE} elementi ripetuti. Di conseguenza diventa necessario l'utilizzo della CHIAVE PRIMARIA. 
        \item Non e' obbligatorio avere righe ordinate nella tabella. Non e' quindi necessario avere nessun vincolo costruttivo.
        \item Ogni attributo deve rappresentare informazioni \textbf{elementari/atomiche}. \newline
        \begin{enumerate}
            \item Se un attributo e' composto allora sara' necessario scomporlo;
            \item Se un campo contiene piu' elementi allora e' necessario creare una nuova tabella connessa alla prima tramite CHIAVE ESTERNA
        \end{enumerate}
    \end{enumerate}
    
   \section{Operazioni di Algebra Lineare}
   \subsection{Sigma - Decomposizione Orizzontale}
   \textit{Corrispettivo dell'operazione di SELECT}
   \paragraph{Sigma in R \\ Applico l'operazione di selezione su una relazione ed ottengo una nuova relazione R}

   \subsection{PiGreco - Decomposizione Verticale}
   \textit{Corrispettivo dell'operazione di SELECT DISTINCT}
   \paragraph{PiGreco in R \\ Applico una proiezione su una relazione R ed ottengo una relazione R} 
   
   \subsection{Theta/Equal - Join}
   \textit{Corrispettivo dell'unione di due tabelle}
   
   \subsection{Left/Inner/Outer/Right Join}
   \textit{Altri differenti tipi di join, considerabili in base all'ordine di lettura della tabella.}
   
   \section{Integrita' dei dati}
   I dati devono essere:
   \subsection{I} 
   \begin{itemize}
       \item \textit{Validi e consistenti}: chiavi usate in maniera corretta.
       \item \textit{Integrita' referenziale}: la coerenza tra le tabelle accoppiate con una chiave primaria/secondaria.
   \end{itemize}
   \subsection{II Vincoli}
   \begin{itemize}
       \item \textit{Entita'}
       \begin{itemize}
           \item Univocita' dell'enupla, esiste ed e' unica.
           \item Vincoli di dominio, tipi di dati corretti secondo vincoli di dominio specifici.
           \item Vincoli di tuple.
       \end{itemize}
       \item \textit{Integrita' referenziale}
       \begin{itemize}
           \item Per ogni valore della chiave esterna, qualora diversa da NULL, deve esistere un valore di PK nella tabella corrispondente.
       \end{itemize}
   \end{itemize}
   \subsection{III Cosa arreca danni?}
   \begin{itemize}
       \item Inserimento
       \item Modifica
       \item Cancellazione
   \end{itemize}
   
   \subsection{IV Regole \textit{Inserimento}}
   \begin{itemize}
       \item Dipendente
       \item Automatico
       \item Nullo
       \item Di Default
       \item Sempre Permesso
   \end{itemize}
  
   \subsection{V Regole \textit{Modifica}}
   \begin{itemize}
       \item Con Restrizione
       \item A Cascata
       \item Nullo
       \item Di Default
       \item Sempre Permesso
   \end{itemize}
   
   \subsection{VI Regole \textit{Cancellazione}}
   \begin{itemize}
       \item Con Restrizione
       \item A Cascata
       \item Nullo
       \item Di Default
       \item Sempre Permesso
   \end{itemize}
   
   \section{Dipendenze, Anomalie, Normalizzazione}
   \subsection{Dipendenze Funzionali}
   \begin{center}
       Data una relazione \textbf{\underline{R}}, una colonna \textbf{\underline{Y}} della relazione,\\ si dice \textbf{funzionalmente dipendente dalla colonna \underline{X}} \\ SE dato un valore di \textbf{\underline{X}} si risce a determinare \textbf{\underline{Y}}.\\
       X \( \rightarrow \) (implica) Y \\
    \end{center} 
    
    Il concetto DESCRIVE \underline{dei legami di tipo funzionale} tra attributi di una stessa relazione. \\
   \begin{enumerate}
       \item cognome
       \item nome
       \item indirizzo
       \item \textbf{codice fiscale} \(\leftarrow\) dipendenza \textbf{funzionale}.\\ Riesco a determinare tutti gli altri valori. 
   \end{enumerate}
   
   Le dipendenze funzionali \textit{implicano} ridondanza, perdendo cosi' la \textit{coerenza} dei dati.
   
   \subsection{Anomalie, con operazioni di manipolazione.}
   \subsubsection{INSERIMENTO}
   \begin{enumerate}
       \item Ridondanza
       \item Attesa per l'inserimento non sapendo tutti i dati corretti
   \end{enumerate}
   \subsubsection{CANCELLAZIONE}
   \begin{enumerate}
       \item Se cancellando un record sono costretto a cancellare ulteriori informazioni correlate
   \end{enumerate}
   \subsubsection{AGGIORNAMENTO}
   \begin{enumerate}
       \item Se aggiornando un record sono costretto ad aggiornare altrettanti record correlati
   \end{enumerate}
   \subsection{Normalizzazione}
   \subsection{Prima Forma Normale -\textit{ Vincoli di N-Uple}}
   \begin{enumerate}
       \item Tutte le n-uple devono avere lo stesso numero di elementi
       \item Per ogni i-esimo elemento di un n-upla vi sono valori omogenei appartenenti allo stesso dominio
      \item R deve avere elementi tutti diversi, non ripetuti
      \item Le n-uple non devono essere necessariamente ordinate
      \item Gli attributi devono rappresentare informazioni atomiche
      \begin{enumerate}
          \item Gli attributi devono rappresentare informazioni elemantari
          \item Se un attributo contiene piu' informazioni, deve essere creata una nuova relazione\textit{ (Chiave Esterna)}
      \end{enumerate}
   \end{enumerate}
   \subsubsection{Seconda Forma Normale - \textit{Chiave Esterna}}
   \begin{enumerate}
       \item \textit{Deve} essere in \textbf{Prima Normale} 
       \item I suoi attributi \textit{non chiave} dipendono dalla \textit{chiave composta intera}
   \end{enumerate}
   \subsubsection{Terza Forma Normale - \textit{Dipendenze Transitive}}
   \begin{enumerate}
       \item \textit{Deve} essere in \textbf{Seconda Normale}
       \item I suoi attributi \textit{non chiave} dipendono dalla \textit{chiave composta intera}
       \item \(\forall\) dipendenza funzionale \textit{X - Y} \textbf{almeno} una delle seguenti condizioni e' verificata
       \begin{itemize}
           \item X  \textbf{contiene} una chiave
           \item Ogni attributo Y \textbf{e' contenuto} in una chiave
       \end{itemize}
   \end{enumerate}

\subsubsection{Forma Normale di Boyce - Codd - \textit{Non sempre raggiungibile}}
\begin{enumerate}
    \item \textit{Deve} essere in \textbf{Seconda Normale}
    \item \(\forall\) dipendenza funzionale \textit{X} e' una chiave \textit{superchiave}
\end{enumerate}
\end{document}